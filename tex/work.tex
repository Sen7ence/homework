\documentclass[12pt,a4paper]{article}
\usepackage{ctex}
\usepackage{geometry}
\usepackage{amsmath,amssymb}
\usepackage{graphicx}
\usepackage{float}
\usepackage{listings}
\usepackage{xcolor}
\usepackage{multirow}
\graphicspath{{../main/figs/}}

% 定义颜色
\definecolor{codegreen}{rgb}{0,0.6,0}
\definecolor{codegray}{rgb}{0.5,0.5,0.5}
\definecolor{codepurple}{rgb}{0.58,0,0.82}
\definecolor{backcolour}{rgb}{0.95,0.95,0.92}

\lstset{
  language=Python,
  basicstyle=\ttfamily\small,
  backgroundcolor=\color{backcolour},   % 背景颜色
  commentstyle=\color{codegreen},       % 注释颜色
  keywordstyle=\color{magenta},         % 关键字颜色
  numberstyle=\tiny\color{codegray},    % 行号颜色
  stringstyle=\color{codepurple},       % 字符串颜色
  frame=single,                         % 在代码周围画框
  rulecolor=\color{black},              % 框颜色
  breaklines=true,
  columns=fullflexible,
  showstringspaces=false,
  numbers=left,                         % 显示行号
  numbersep=5pt,                        % 行号间距
  tabsize=2,                            % tab大小
  captionpos=b                          % 标题位置
}
\geometry{left=2.5cm,right=2.5cm,top=2.5cm,bottom=2.5cm}

\title{高等工程热力学编程部分作业}
\author{何飏 \quad 3123101186}
\date{}
\begin{document}

\maketitle

采用Python语言进行计算程序编写,将PR方程相关计算封装在PR类中,程序默认要求输入两种流体性质,对于同种物质,输入相同参数。对于不同的问题在if \_\_name\_\_ == "\_\_main\_\_"部分进行编辑以进行计算。计算思路如下:

(1)导入流体的相关性质如临界温度$T_\mathrm{c}$、临界压力$p_\mathrm{c}$、偏心因子$\omega$和摩尔质量$M$,以及理想气体常数$R=8.314462618\mathrm{J/(mol\cdot K)}$;

(2)给定温度$T$计算参数$a$、$b$和$a'$,其中$a'$根据书中式(3.110)给出的混合法则对$T$进行求导,求导结果如下:
\begin{equation*}
    a'=x_1^2 a_1' + x_1 x_2 (1-k_{ij})(a_1'\sqrt{\frac{a_2}{a_1}} + a_2'\sqrt{\frac{a_1}{a_2}}) + x_2^2 a_2'
\end{equation*}

(3)给定温度$T$和压力$p$计算参数$A$和$B$;

(4)计算压缩因子$Z$的多项式系数$C_2, C_1, C_0$;

(5)使用牛顿法分别计算压缩因子$Z$的液相值$Z_l$和气相值$Z_g$;

(6)计算比体积$v$的液相值$v_l$和气相值$v_g$;

(7)绘制比体积$v$与温度$T$的关系曲线,并标注饱和温度$T_\mathrm{sat}$;

(8)计算焓的余函数$h^\mathrm{r}$和熵的余函数$s^\mathrm{r}$;

(9)计算热容$c_p$的积分和热容除以温度$\frac{c_p}{T}$的积分;

(10)计算焓$h$和熵$s$;

(11)计算气相的逸度系数$\hat{\phi}$和逸度$\hat{f}$;

(12)绘制气相的逸度系数$\hat{\phi}$和逸度$\hat{f}$与温度$T$的关系曲线。

计算所用的PR类程序如下:
\begin{lstlisting}
import numpy as np
import matplotlib.pyplot as plt
import os

# 使用 Times New Roman 作为 matplotlib 全局字体
plt.rcParams["font.family"] = "serif"
plt.rcParams["font.serif"] = ["Times New Roman"]
plt.rcParams["mathtext.fontset"] = "stix"

class PR:
    def __init__(self, Tc1, pc1, omega1, M1, x1, Tc2, pc2, omega2, M2, kij, ps0):
        self.Tc1 = Tc1  # K
        self.pc1 = pc1 * 1e6  # Pa,输入MPa
        self.omega1 = omega1  # 无量纲
        self.M1 = M1 / 1e3  # kg/mol,输入g/mol
        self.x1 = x1  # 组分1的摩尔分数

        self.Tc2 = Tc2  # K
        self.pc2 = pc2 * 1e6  # Pa,输入MPa
        self.omega2 = omega2  # 无量纲
        self.M2 = M2 / 1e3  # kg/mol,输入g/mol
        self.x2 = 1 - x1  # 组分2的摩尔分数

        self.ps0 = ps0  # MPa

        self.kij = kij  # 无量纲

    R = 8.314462618  # J/(mol*K)

    # 计算a和b
    def params(self, T):
        kappa1 = 0.37464 + 1.54226 * self.omega1 - 0.26992 * self.omega1**2
        kappa2 = 0.37464 + 1.54226 * self.omega2 - 0.26992 * self.omega2**2
        Tr1 = T / self.Tc1
        Tr2 = T / self.Tc2
        alpha1 = (1 + kappa1 * (1 - Tr1**0.5)) ** 2
        alpha2 = (1 + kappa2 * (1 - Tr2**0.5)) ** 2
        a1 = 0.45724 * self.R**2 * self.Tc1**2 / self.pc1 * alpha1
        a2 = 0.45724 * self.R**2 * self.Tc2**2 / self.pc2 * alpha2
        da1 = (
            -0.45724
            * self.R**2
            * self.Tc1**2
            / self.pc1
            * kappa1
            * (1 + kappa1 * (1 - Tr1**0.5))
            * (Tr1**-0.5)
            / self.Tc1
        )
        da2 = (
            -0.45724
            * self.R**2
            * self.Tc2**2
            / self.pc2
            * kappa2
            * (1 + kappa2 * (1 - Tr2**0.5))
            * (Tr2**-0.5)
            / self.Tc2
        )
        b1 = 0.07780 * self.R * self.Tc1 / self.pc1
        b2 = 0.07780 * self.R * self.Tc2 / self.pc2

        a = (
            self.x1**2 * a1
            + self.x2**2 * a2
            + 2 * self.x1 * self.x2 * (a1 * a2) ** 0.5 * (1 - self.kij)
        )
        b = self.x1 * b1 + self.x2 * b2
        da = (
            self.x1**2 * da1
            + self.x2**2 * da2
            + self.x1
            * self.x2
            * (1 - self.kij)
            * ((a2 / a1) ** 0.5 * da1 + (a1 / a2) ** 0.5 * da2)
        )
        return a1, a2, a, b1, b2, b, da

    # 计算A和B
    def AB(self, T, p):
        a1, a2, a, b1, b2, b, da = self.params(T)
        A = a * p * 1e6 / (self.R * T) ** 2
        B = b * p * 1e6 / (self.R * T)
        return A, B

    # 计算C2, C1, C0
    def C(self, T, p):
        A, B = self.AB(T, p)
        C2 = -(1 - B)
        C1 = A - 3 * B**2 - 2 * B
        C0 = -(A * B - B**2 - B**3)
        return C2, C1, C0

    # 计算压缩因子Z
    # 液相
    def Zl(self, T, p):
        C2, C1, C0 = self.C(T, p)
        # 牛顿法求解Z
        Zl = 0.001  # 初始猜测值
        for _ in range(100):
            f = Zl**3 + C2 * Zl**2 + C1 * Zl + C0
            df = 3 * Zl**2 + 2 * C2 * Zl + C1
            Zl_new = Zl - f / df
            if abs(Zl_new - Zl) < 1e-6:
                break
            Zl = Zl_new
        return Zl

    # 气相
    def Zg(self, T, p):
        C2, C1, C0 = self.C(T, p)
        # 牛顿法求解Z
        Zg = 1.0  # 初始猜测值
        for _ in range(100):
            f = Zg**3 + C2 * Zg**2 + C1 * Zg + C0
            df = 3 * Zg**2 + 2 * C2 * Zg + C1
            Zg_new = Zg - f / df
            if abs(Zg_new - Zg) < 1e-6:
                break
            Zg = Zg_new
        return Zg

    # 计算比体积v
    # 液相
    def vl(self, T, p):
        Zl = self.Zl(T, p)
        vl = Zl * self.R * T / (p * 1e6)
        return vl  # m³/mol

    # 气相
    def vg(self, T, p):
        Zg = self.Zg(T, p)
        vg = Zg * self.R * T / (p * 1e6)
        return vg  # m³/mol

    def plot_Tv(
        self,
        fluid_name,  # 流体名称
        p,  # 压力 MPa
        Tsat,  # 饱和温度 K
        T_min,  # 温度范围最小值 K
        T_max,  # 温度范围最大值 K
        nT=220,  # 温度点数
    ):
        T_grid = np.linspace(T_min, T_max, nT)  # 温度网格
        v_grid = np.empty_like(T_grid)  # 比体积网格
        # 计算比体积
        for i, T in enumerate(T_grid):
            if T < Tsat:
                v_grid[i] = self.vl(T, p)
            elif T > Tsat:
                v_grid[i] = self.vg(T, p)
            else:
                v_grid[i] = 0.5 * (self.vl(T, p) + self.vg(T, p))
        fig, ax = plt.subplots()  # 创建图像和坐标轴
        # 主曲线
        ax.plot(
            v_grid,
            T_grid,
            linewidth=2,
            label=fluid_name,
        )
        xmin, xmax = np.nanmin(v_grid), np.nanmax(v_grid)
        # Tsat 虚线
        ax.hlines(Tsat, xmin, xmax, linestyles="--", label=r"$T_{\mathrm{sat}}$")
        # 标注 Tsat
        yt = list(ax.get_yticks())
        # 加入Tsat并排序
        if not any(abs(t - Tsat) < 1e-8 for t in yt):
            yt.append(Tsat)
        yt = np.array(sorted(yt))
        # 生成刻度标签:对 Tsat 使用仅数值标签(两位小数),其它刻度保留数字格式(根据范围选择小数位)
        deltaT = T_grid.max() - T_grid.min()
        labels = []
        for t in yt:
            if abs(t - Tsat) < 1e-8 or abs(t - Tsat) < 1e-6 * max(1.0, deltaT):
                labels.append(f"{Tsat:.2f}")
            else:
                # 根据温度范围决定格式,避免过多小数
                if deltaT > 50:
                    labels.append(f"{t:.0f}")
                else:
                    labels.append(f"{t:.2f}")
        ax.set_yticks(yt)
        ax.set_yticklabels(labels)
        # 轴标签
        ax.set_xlabel(r"$v$ (m³/mol)")
        ax.set_ylabel(r"$T$ (K)")
        # 标题
        ax.set_title(f"{fluid_name}  $v$–$T$ at $p$ = {p:.1f} MPa")
        ax.grid(True)
        ax.set_xscale("log")  # 使用对数刻度
        ax.legend(loc="upper left", frameon=True, fancybox=True, framealpha=0.9)

        # 固定保存路径为脚本同目录下的 figs 文件夹
        base_dir = os.path.dirname(os.path.abspath(__file__))
        fig_dir = os.path.join(base_dir, "figs")
        os.makedirs(fig_dir, exist_ok=True)

        # 文件名固定为"流体名称.png"
        filename = f"{fluid_name}.png"
        savepath = os.path.join(fig_dir, filename)

        # 保存图像,固定参数
        fig.savefig(savepath, dpi=300, bbox_inches="tight", transparent=False)
        plt.close(fig)

    # 计算焓的余函数
    # 液相
    def h_res_l(self, T, p):
        a1, a2, a, b1, b2, b, da = self.params(T)
        Zl = self.Zl(T, p)
        vl = self.vl(T, p)
        hr_l = (T * da - a) / (b * np.sqrt(8)) * np.log(
            (vl - 0.414 * b) / (vl + 2.414 * b)
        ) + self.R * T * (1 - Zl)
        return hr_l

    # 气相
    def h_res_g(self, T, p):
        a1, a2, a, b1, b2, b, da = self.params(T)
        Zg = self.Zg(T, p)
        vg = self.vg(T, p)
        hr_g = (T * da - a) / (b * np.sqrt(8)) * np.log(
            (vg - 0.414 * b) / (vg + 2.414 * b)
        ) + self.R * T * (1 - Zg)
        return hr_g

    # 计算熵的余函数
    # 液相
    def s_res_l(self, T, p):
        a1, a2, a, b1, b2, b, da = self.params(T)
        vl = self.vl(T, p)
        sr_l = (
            -self.R * np.log((vl - b) / vl)
            - self.R * np.log(vl / (self.R * T / (p * 1e6)))
            + da / (b * np.sqrt(8)) * np.log((vl - 0.414 * b) / (vl + 2.414 * b))
        )
        return sr_l

    # 气相
    def s_res_g(self, T, p):
        a1, a2, a, b1, b2, b, da = self.params(T)
        vg = self.vg(T, p)
        sr_g = (
            -self.R * np.log((vg - b) / vg)
            - self.R * np.log(vg / (self.R * T / (p * 1e6)))
            + da / (b * np.sqrt(8)) * np.log((vg - 0.414 * b) / (vg + 2.414 * b))
        )
        return sr_g

    # 计算c_p积分
    def cp(self, T, A, B, C, D):
        cp = (
            A * (T - 273.15)
            + B / 2 * (T**2 - 273.15**2)
            + C / 3 * (T**3 - 273.15**3)
            + D / 4 * (T**4 - 273.15**4)
        )
        return cp

    # 计算c_p/T积分
    def cpT(self, T, A, B, C, D):
        cpT = (
            A * np.log(T / 273.15)
            + B * (T - 273.15)
            + C / 2 * (T**2 - 273.15**2)
            + D / 3 * (T**3 - 273.15**3)
        )
        return cpT

    # 计算焓和熵
    # 液相
    def h_l(self, T, A, B, C, D, p):
        h_r_ps_0 = self.h_res_l(273.15, self.ps0)
        cp0 = self.cp(T, A, B, C, D)
        h_res_l = self.h_res_l(T, p)
        hl = (
            200 * 1e3
            + cp0
            + (h_r_ps_0 - h_res_l) / (self.x1 * self.M1 + self.x2 * self.M2)
        )  # J/kg
        return hl

    def s_l(self, T, A, B, C, D, p):
        s_r_ps_0 = self.s_res_l(273.15, self.ps0)
        cpT = self.cpT(T, A, B, C, D)
        sr_l = self.s_res_l(T, p)
        sl = (
            1e3
            + cpT
            + (s_r_ps_0 - self.R * np.log(p / self.ps0) - sr_l)
            / (self.x1 * self.M1 + self.x2 * self.M2)
        )  # J/(kg*K)
        return sl

    # 气相
    def h_g(self, T, A, B, C, D, p):
        h_r_ps_0 = self.h_res_l(273.15, self.ps0)  # 使用液相作为基准
        cp0 = self.cp(T, A, B, C, D)
        h_res_g = self.h_res_g(T, p)
        hg = (
            200 * 1e3
            + cp0
            + (h_r_ps_0 - h_res_g) / (self.x1 * self.M1 + self.x2 * self.M2)
        )  # J/kg
        return hg

    def s_g(self, T, A, B, C, D, p):
        s_r_ps_0 = self.s_res_l(273.15, self.ps0)  # 使用液相作为基准
        cpT = self.cpT(T, A, B, C, D)
        sr_g = self.s_res_g(T, p)
        sg = (
            1e3
            + cpT
            + (s_r_ps_0 - self.R * np.log(p / self.ps0) - sr_g)
            / (self.x1 * self.M1 + self.x2 * self.M2)
        )  # J/(kg*K)
        return sg

    def phi(self, T, p):
        Zg = self.Zg(T, p)
        a1, a2, a, b1, b2, b, da = self.params(T)
        A, B = self.AB(T, p)
        phi1 = np.exp(
            (b1 / b) * (Zg - 1)
            - np.log(Zg - B)
            - A
            / (B * np.sqrt(8))
            * (2 * (self.x2 * (a1 * a2) ** 0.5 + self.x1 * a1) / a - b1 / b)
            * np.log((Zg + 2.414 * B) / (Zg - 0.414 * B))
        )
        phi2 = np.exp(
            (b2 / b) * (Zg - 1)
            - np.log(Zg - B)
            - A
            / (B * np.sqrt(8))
            * (2 * (self.x1 * (a1 * a2) ** 0.5 + self.x2 * a2) / a - b2 / b)
            * np.log((Zg + 2.414 * B) / (Zg - 0.414 * B))
        )
        return phi1, phi2

    def f(self, T, p):  # MPa
        phi1, phi2 = self.phi(T, p)
        f1 = self.x1 * phi1 * p
        f2 = self.x2 * phi2 * p
        return f1, f2
    
    if __name__ == "__main__":
        pass
\end{lstlisting}

\newpage

\subsection*{3-10}

R290在1.4MPa下的$T$-$v$图和R600a在0.6MPa下的$T$-$v$图如下:
\begin{figure}[H]
    \centering
    \includegraphics[width=0.7\textwidth]{R290.png}
    \caption{R290在1.4MPa下的$T$-$v$图}
\end{figure}

\begin{figure}[H]
    \centering
    \includegraphics[width=0.7\textwidth]{R600a.png}
    \caption{R600a在0.6MPa下的$T$-$v$图}
\end{figure}

计算程序如下:

\begin{lstlisting}
    R290 = PR(
        Tc1=369.89,  # K
        pc1=4.2512,  # MPa
        omega1=0.1521,  # 无量纲
        M1=44.096,  # g/mol
        x1=1.0,
        Tc2=369.89,
        pc2=4.2512,
        omega2=0.1521,
        M2=44.096,
        kij=0.064,
        ps0=0.47446,
    )

    R600a = PR(
        Tc1=407.81,
        pc1=3.629,
        omega1=0.184,
        M1=58.122,  # R600a
        x1=1.0,
        Tc2=407.81,
        pc2=3.629,
        omega2=0.184,
        M2=58.122,  # R600a
        kij=0.0,
        ps0=0.15696,
    )

    R290.plot_Tv("R290", 1.4, 317.86, 200, 450)
    R600a.plot_Tv("R600a", 0.6, 314.12, 200, 450)
\end{lstlisting}

% \newpage

\subsection*{3-13}
查物性库得,对于R134a,各参数为:$T_\mathrm{c}=374.21\mathrm{K}$,$p_\mathrm{c}=4.0593\mathrm{MPa}$,$\omega=0.326$,$M=102.03\mathrm{g/mol}$。

对于R1234yf,各参数为:$T_\mathrm{c}=367.85\mathrm{K}$,$p_\mathrm{c}=3.3822\mathrm{MPa}$,$\omega=0.276$,$M=114.04\mathrm{g/mol}$;

对于R1234ze(E),各参数为:$T_\mathrm{c}=382.75\mathrm{K}$,$p_\mathrm{c}=3.6349\mathrm{MPa}$,$\omega=0.313$,$M=114.04\mathrm{g/mol}$;

压力为0.1MPa,温度为35℃=308.15K时,以上三种制冷剂均为气相,利用程序进行计算$v_g$。
计算结果为:
$v_\mathrm{R134a}=0.24679\mathrm{m^3/kg}$,$v_\mathrm{R1234yf}=0.22031\mathrm{m^3/kg}$,$v_\mathrm{R1234ze(E)}=0.22006\mathrm{m^3/kg}$

可以看出,三种制冷剂的比体积相差不大,R134a的比体积略大于另外两种,故采用R1234yf和R1234ze(E)作为R134a的替代品是合理的。

计算程序为:
\begin{lstlisting}
        R134a = PR(
        Tc1=374.21,
        pc1=4.0593,
        omega1=0.326,
        M1=102.03,  # R134a
        x1=1.0,
        Tc2=374.21,
        pc2=4.0593,
        omega2=0.326,
        M2=102.03,  # R134a
        kij=0.0,
        ps0=0.57245,
    )
    R1234yf = PR(
        Tc1=367.85,
        pc1=3.3822,
        omega1=0.276,
        M1=114.04,  # R1234yf
        x1=1.0,
        Tc2=367.85,
        pc2=3.3822,
        omega2=0.276,
        M2=114.04,  # R1234yf
        kij=0.0,
        ps0=0.42483,
    )
    R1234ze = PR(
        Tc1=382.75,
        pc1=3.6349,
        omega1=0.313,
        M1=114.04,  # R1234ze
        x1=1.0,
        Tc2=382.75,
        pc2=3.6349,
        omega2=0.313,
        M2=114.04,  # R1234ze
        kij=0.0,
        ps0=0.49314,
    )

print(R134a.vg(308.15, 0.1))
print(R1234yf.vg(308.15, 0.1))
print(R1234ze.vg(308.15, 0.1))
\end{lstlisting}

\subsection*{3-15}

在压力$p=0.1\mathrm{MPa}$、$0.2\mathrm{MPa}$、$0.3\mathrm{MPa}$,温度$T=300\mathrm{K}$时,不同的$k_{ij}$条件下,混合制冷剂R290/R600a的比体积计算结果与计算偏差如表1所示,表中计算偏差是相对于$k_{ij}=0.064$时的比体积计算结果而言的。可以看出,$k_{ij}$取0.1、0和-0.1时,计算结果与$k_{ij}=0.064$时的比体积计算结果偏差逐渐增大,且偏差均小于1\%。
\begin{table}[h]
\centering
\begin{tabular}{c c c c}
\hline
$p$ (MPa) & $k_{ij}$ &  $v$ (m³/mol) &误差 (\%) \\
\hline
\multirow{4}{*}{0.1} & 0.064 & 0.47838 &   \\
 & 0.1 & 0.47859 & 0.04390\\
 & 0 & 0.47802 & 0.07525  \\
 & -0.1 & 0.47744 & 0.19650  \\
\hline
\multirow{4}{*}{0.2} & 0.064 & 0.23422 &   \\
 & 0.1 & 0.23443 & 0.08966  \\
 & 0 & 0.23384 & 0.16224  \\
 & -0.1 & 0.23324 & 0.41841  \\
\hline
\multirow{4}{*}{0.3} & 0.064 & 0.15273 &   \\
 & 0.1 & 0.15295 & 0.14405  \\
 & 0 & 0.15233 & 0.26190  \\
 & -0.1 & 0.15171 & 0.66785  \\
\hline
\end{tabular}
\caption{不同$k_{ij}$条件下混合制冷剂R290/R600a的比体积计算结果与计算偏差}
\end{table}

代码如下:

\begin{lstlisting}
    # kij = 0.064, p=0.1MPa情况下代码
    R290R600a = PR(
        Tc1=369.89,
        pc1=4.2512,
        omega1=0.1521,
        M1=44.096,  # R290
        x1=0.5,
        Tc2=407.81,
        pc2=3.629,
        omega2=0.184,
        M2=58.122,  # R600a
        kij=0.064,
        ps0=0.32979,
    )

    print(
        R290R600a.vg(300, 0.1)
        / (R290R600a.x1 * R290R600a.M1 + R290R600a.x2 * R290R600a.M2)
    )
\end{lstlisting}

\newpage

\section*{第四章}

\subsection*{4-13}

利用程序分别计算在1.4MPa下不同温度$T$下R290的液相焓和熵,以及在0.6MPa下不同温度$T$下R600a的液相焓和熵,计算结果与标准值对比如表2和表3所示,可以看出,计算结果与标准值误差均小于1\%。

\begin{table}[h]
\centering
\caption{1.4MPa下不同温度$T$下R290的液相焓和熵计算结果与标准值对比}
\begin{tabular}{c c c c c c c}
\hline
$T$ (K) & $h$ (kJ/kg) &$h_\mathrm{标准}$(kJ/kg) & $s$ (Jk/(kg·K)) &$s_\mathrm{标准}$(kJ/(kg·K))& $h$误差\%&$s$误差\%\\
\hline
260 & 168.686& 170.083& 0.876& 0.881& 0.821& 0.567\\
270 & 192.542& 194.346& 0.966& 0.973& 0.928& 0.719\\
280 & 217.443& 219.262& 1.057& 1.063& 0.830& 0.723\\
290 & 243.544& 244.914& 1.149& 1.153& 0.559& 0.347\\
300 & 271.043& 271.376& 1.242& 1.243& 0.123& 0.080\\
\hline
\end{tabular}
\end{table}

\begin{table}[h]
\centering
\caption{0.6MPa下不同温度$T$下R600a的液相焓和熵计算结果与标准值对比}
\begin{tabular}{c c c c c c c}
\hline
$T$ (K) & $h$ (kJ/kg) &$h_\mathrm{标准}$(kJ/kg) & $s$ (kJ/(kg·K)) &$s_\mathrm{标准}$(kJ/(kg·K))& $h$误差\%&$s$误差\%\\
\hline
260 & 171.745& 171.556& 0.891& 0.891& 0.110& 0\\
270 & 193.349& 193.946& 0.973& 0.975& 0.308& 0.205\\
280 & 215.677& 216.839& 1.054& 1.058& 0.536& 0.378\\
290 & 238.773& 240.279& 1.135& 1.140& 0.627& 0.438\\
300 & 262.694& 264.277& 1.216& 1.222& 0.158& 0.491\\
\hline
\end{tabular}
\end{table}

程序如下:
\begin{lstlisting}
    R290 = PR(
        Tc1=369.89,  # K
        pc1=4.2512,  # MPa
        omega1=0.1521,  # 无量纲
        M1=44.096,  # g/mol
        x1=1.0,
        Tc2=369.89,
        pc2=4.2512,
        omega2=0.1521,
        M2=44.096,
        kij=0.064,
        ps0=0.47446,
    )

    R600a = PR(
        Tc1=407.81,
        pc1=3.629,
        omega1=0.184,
        M1=58.122,  # R600a
        x1=1.0,
        Tc2=407.81,
        pc2=3.629,
        omega2=0.184,
        M2=58.122,  # R600a
        kij=0.0,
        ps0=0.15696,
    )
    # 300K下结果
    print(R290.h_l(300, -95.80, 6.945, -3.597 * 1e-3, 7.290 * 1e-7, 1.4))
    print(R290.s_l(300, -95.80, 6.945, -3.597 * 1e-3, 7.290 * 1e-7, 1.4))
    print(R600a.h_l(300, -23.91, 6.605, -3.176 * 1e-3, 4.981 * 1e-7, 0.6))
    print(R600a.s_l(300, -23.91, 6.605, -3.176 * 1e-3, 4.981 * 1e-7, 0.6))
\end{lstlisting}

\subsection*{4-15}
取二元作用系数$k_{ij}=0.064$,在$p$=1.0MPa下不同温度下计算R290/R600a(50\%/50\%)混合制冷剂的焓和熵,计算结果如表4所示,结果表明,计算结果与标准值误差很小。

\begin{table}[h]
\centering
\caption{1.0MPa下不同温度$T$下R290/R600a(50\%/50\%)混合制冷剂的液相焓和熵计算结果与标准值对比}
\begin{tabular}{c c c c c c c}
\hline
$T$ (K) & $h$ (kJ/kg) &$h_\mathrm{标准}$(kJ/kg) & $s$ (kJ/(kg·K)) &$s_\mathrm{标准}$(kJ/(kg·K))& $h$误差\%&$s$误差\%\\
\hline
260 & 170.449& 170.056& 0.885& 0.889& 0.231& 0.450\\
270 & 193.011& 193.144& 0.970& 0.979& 0.069& 0.919\\
280 & 216.459& 216.813& 1.055& 1.066& 0.163& 1.032\\
290 & 240.887& 241.124& 1.141& 1.150& 0.098& 0.783\\
300 & 266.430& 266.154& 1.228& 1.231& 0.103& 0.244\\
\hline
\end{tabular}
\end{table}

程序如下:
\begin{lstlisting}
    R290R600a = PR(
        Tc1=369.89,
        pc1=4.2512,
        omega1=0.1521,
        M1=44.096,  # R290
        x1=0.5,
        Tc2=407.81,
        pc2=3.629,
        omega2=0.184,
        M2=58.122,  # R600a
        kij=0.064,
        ps0=0.32979,
    )
    # 300K下结果
    print(R290R600a.h_l(260, -59.81, 6.775, -3.386 * 1e-3, 6.135 * 1e-7, 1))
    print(R290R600a.s_l(260, -59.81, 6.775, -3.386 * 1e-3, 6.135 * 1e-7, 1))
\end{lstlisting}

\section*{第六章}
\subsection*{6-11}

推导过程见作业,1.4MPa下R290/R600a混合制冷剂的$\hat{\phi}$-$T$图和$\hat{f}$-$T$图如下:
\begin{figure}[H]
    \centering
    \includegraphics[width=1\textwidth]{R290R600a_f.png}
    \caption{1.4MPa下R290/R600a混合制冷剂的$\hat{\phi}$-$T$图和$\hat{f}$-$T$}
\end{figure}

程序如下:
\begin{lstlisting}
    R290R600a = PR(
        Tc1=369.89,
        pc1=4.2512,
        omega1=0.1521,
        M1=44.096,  # R290
        x1=0.5,
        Tc2=407.81,
        pc2=3.629,
        omega2=0.184,
        M2=58.122,  # R600a
        kij=0.064,
        ps0=0.32979,
    )
    R290R600a.plot_fT("R290R600a", 1.4, 350, 450, 11)
\end{lstlisting}

\end{document}