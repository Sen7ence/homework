\documentclass[12pt,a4paper]{article}
\usepackage{ctex}
\usepackage{geometry}
\usepackage{amsmath,amssymb}
\usepackage{graphicx}
\usepackage{listings}
\usepackage{xcolor}
\lstset{
  language=Python,
  basicstyle=\ttfamily\small,
  frame=single,             % 在代码周围画框
  rulecolor=\color{black},  % 框颜色
  breaklines=true,
  columns=fullflexible,
  showstringspaces=false
}
\geometry{left=2.5cm,right=2.5cm,top=2.5cm,bottom=2.5cm}

\title{高等工程热力学\\第3章作业}
\author{姓名:\underline{\hspace{4cm}} \quad 学号:\underline{\hspace{4cm}}}
\date{\today}

\begin{document}
\renewcommand{\maketitle}{}
\maketitle

\section*{第三章}

\subsection*{3-10}
采用Python进行计算机程序编写,计算程序如下:
\begin{lstlisting}
import numpy as np
import matplotlib.pyplot as plt
import os

# 使用 Times New Roman 作为 matplotlib 全局字体
plt.rcParams["font.family"] = "serif"
plt.rcParams["font.serif"] = ["Times New Roman"]
plt.rcParams["mathtext.fontset"] = "stix"

class PRFluid:
    def __init__(self, Tc, Pc, omega, M):
        self.Tc = Tc  # K
        self.Pc = Pc  # Pa
        self.omega = omega  # 无量纲
        self.M = M  # kg/mol

    R = 8.314462618  # J/(mol*K)

    # 计算a和b
    def params(self, T):
        kappa = 0.37464 + 1.54226 * self.omega - 0.26992 * self.omega**2
        Tr = T / self.Tc
        alpha = (1 + kappa * (1 - Tr**0.5)) ** 2
        a = 0.45724 * self.R**2 * self.Tc**2 / self.Pc * alpha
        b = 0.07780 * self.R * self.Tc / self.Pc
        return a, b

    # 计算A和B
    def AB(self, T, P):
        a, b = self.params(T)
        A = a * P / (self.R * T) ** 2
        B = b * P / (self.R * T)
        return A, B

    # 计算C2, C1, C0
    def C(self, T, P):
        A, B = self.AB(T, P)
        C2 = -(1 - B)
        C1 = A - 3 * B**2 - 2 * B
        C0 = -(A * B - B**2 - B**3)
        return C2, C1, C0

    # 计算压缩因子Z
    # 液相
    def Zl(self, T, P):
        C2, C1, C0 = self.C(T, P)
        # 牛顿法求解Z
        Zl = 0.001  # 初始猜测值
        for _ in range(100):
            f = Zl**3 + C2 * Zl**2 + C1 * Zl + C0
            df = 3 * Zl**2 + 2 * C2 * Zl + C1
            Zl_new = Zl - f / df
            if abs(Zl_new - Zl) < 1e-6:
                break
            Zl = Zl_new
        return Zl

    # 气相
    def Zg(self, T, P):
        C2, C1, C0 = self.C(T, P)
        # 牛顿法求解Z
        Zg = 1.0  # 初始猜测值
        for _ in range(100):
            f = Zg**3 + C2 * Zg**2 + C1 * Zg + C0
            df = 3 * Zg**2 + 2 * C2 * Zg + C1
            Zg_new = Zg - f / df
            if abs(Zg_new - Zg) < 1e-6:
                break
            Zg = Zg_new
        return Zg

    # 计算比体积v
    # 液相
    def vl(self, T, P):
        Zl = self.Zl(T, P)
        vl = Zl * self.R * T / (P * self.M)
        return vl

    # 气相
    def vg(self, T, P):
        Zg = self.Zg(T, P)
        vg = Zg * self.R * T / (P * self.M)
        return vg

    # 画v-T图
    def plot_Tv(
        self,
        fluid_name,
        P,
        Tsat,
        T_min,
        T_max,
        nT=220,
        savepath=None,
        dpi=300,
        bbox_inches="tight",
        transparent=False,
        close_fig=False,
    ):
        T_grid = np.linspace(T_min, T_max, nT)
        v_grid = np.empty_like(T_grid)
        for i, T in enumerate(T_grid):
            if T < Tsat:
                v_grid[i] = self.vl(T, P)
            elif T > Tsat:
                v_grid[i] = self.vg(T, P)
            else:
                v_grid[i] = 0.5 * (self.vl(T, P) + self.vg(T, P))
        fig, ax = plt.subplots()
        # 主曲线
        ax.plot(v_grid, T_grid, linewidth=2, label=fluid_name)
        xmin, xmax = np.nanmin(v_grid), np.nanmax(v_grid)
        # Tsat 虚线
        ax.hlines(Tsat, xmin, xmax, linestyles="--", label=r"$T_{\mathrm{sat}}$")
        # 用纵轴刻度标注 Tsat
        yt = list(ax.get_yticks())
        # 如果 Tsat 不在当前刻度中,加入并排序
        if not any(abs(t - Tsat) < 1e-8 for t in yt):
            yt.append(Tsat)
        yt = np.array(sorted(yt))
        # 生成刻度标签:对 Tsat 使用仅数值标签(两位小数),其它刻度保留数字格式(根据范围选择小数位)
        deltaT = T_grid.max() - T_grid.min()
        labels = []
        for t in yt:
            if abs(t - Tsat) < 1e-8 or abs(t - Tsat) < 1e-6 * max(1.0, deltaT):
                labels.append(f"{Tsat:.2f}")
            else:
                # 根据温度范围决定格式,避免过多小数
                if deltaT > 50:
                    labels.append(f"{t:.0f}")
                else:
                    labels.append(f"{t:.2f}")
        ax.set_yticks(yt)
        ax.set_yticklabels(labels)
        # 轴标签
        ax.set_xlabel(r"$v$ (m³/kg)")
        ax.set_ylabel(r"$T$ (K)")
        # 标题
        ax.set_title(f"{fluid_name}  $v$–$T$ at $P$ = {P/1e6:.3f} MPa")
        ax.grid(True)
        ax.set_xscale("log")  # 使用对数刻度
        ax.legend(loc="upper left", frameon=True, fancybox=True, framealpha=0.9)
        # 默认保存路径为脚本同目录下的 fig 文件夹(当 savepath 为 None 时)
        if savepath is None:
            base_dir = os.path.dirname(os.path.abspath(__file__))
            fig_dir = os.path.join(base_dir, "fig")
            os.makedirs(fig_dir, exist_ok=True)
            # 简单替换文件名中的非法字符
            safe_name = "".join(
                c if c.isalnum() or c in ("_", "-") else "_" for c in fluid_name
            )
            filename = f"{safe_name}_v-T_P{P/1e6:.3f}MPa.png"
            savepath = os.path.join(fig_dir, filename)

        # 保存图像(如果提供或已生成 savepath)
        if savepath:
            dirname = os.path.dirname(savepath)
            if dirname and not os.path.exists(dirname):
                os.makedirs(dirname, exist_ok=True)
            fig.savefig(
                savepath, dpi=dpi, bbox_inches=bbox_inches, transparent=transparent
            )
            if close_fig:
                plt.close(fig)

        return fig

R290 = PRFluid(369.89, 4.2512e6, 0.1521, 44.096 / 1000)  # kg/mol
R290.plot_Tv("R290", 1.4e6, 317.86, 200, 450, savepath="R290_v-T.png")

R600a = PRFluid(407.81, 3.629e6, 0.184, 58.122 / 1000)  # kg/mol
R600a.plot_Tv("R600a", 0.6e6, 314.12, 200, 450, savepath="R600a_v-T.png")


\end{lstlisting}

\subsection*{3-13}

\subsection*{3-15}

\vspace{1cm}

\section*{第四章}

\subsection*{4-13}
采用Python进行计算机程序编写,计算程序如下:
\begin{lstlisting}

\end{lstlisting}
\subsection*{4-15}

\end{document}